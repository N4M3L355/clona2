\begin{figure}[h]
    \centering
    \begin{tikzpicture}[->, >=latex, auto, semithick, node distance=3cm]
        \tikzstyle{state}=[draw, rectangle, rounded corners, thick, fill=gray!10, minimum size=6mm]
        \tikzstyle{errorstate}=[draw, rectangle, rounded corners, thick, fill=red!20, minimum size=6mm]

        \node[state, initial]    (Disconnected)                {Disconnected};
        \node[state]             (Initializing) [right of=Disconnected] {Initializing};
        \node[state]             (Connected)   [right of=Initializing]  {Connected};
        \node[state]             (Configuring) [above right of=Connected] {Configuring};
        \node[state]             (Capturing)   [below right of=Connected] {Capturing};
        \node[state]             (Processing)  [right of=Capturing] {Processing};
        \node[state, accepting]  (Idle)        [above right of=Processing] {Idle};
        \node[errorstate]        (Error)       [below of=Processing] {Error};

        \path
        (Disconnected) edge node {Start App} (Initializing)
        (Initializing) edge node {Camera Found} (Connected)
        (Initializing) edge [bend left] node {Camera Not Found} (Disconnected)
        (Connected) edge [bend left] node {Disconnect} (Disconnected)
        (Connected) edge [bend left] node {Configure} (Configuring)
        (Configuring) edge [bend left] node {Success} (Connected)
        (Configuring) edge [bend left] node {Failure} (Error)
        (Connected) edge [bend right] node [swap] {Capture} (Capturing)
        (Capturing) edge node {Success} (Processing)
        (Capturing) edge [bend left] node {Failure} (Error)
        (Processing) edge node {Complete} (Idle)
        (Processing) edge [bend left] node {Error} (Error)
        (Error) edge [bend left] node {Recover} (Connected)
        (Error) edge [bend right] node [swap] {Disconnect} (Disconnected)
        (Idle) edge [loop above] node {Wait} (Idle)
        (Idle) edge [bend left] node {New Action} (Connected)
        ;
    \end{tikzpicture}
    \caption{State Diagram of Camera Operation within the Application}
    \label{fig:camera_state_diagram}
\end{figure}


\subsection{Environmental Considerations}
To maximize repeatability despite the DIY nature of the tests, the methodology includes practical guidelines:

\begin{itemize}
    \item \textbf{Lighting}: Testing should be performed in diffuse indoor lighting, avoiding direct sunlight or strong directional light sources that could create uneven illumination.
    \item \textbf{Distance}: Rough distance guidelines based on focal length are provided, emphasizing consistency between tests over absolute precision.
    \item \textbf{Surface Selection}: Guidelines for selecting appropriate test surfaces (e.g., wall smoothness, grid contrast) help users achieve reliable results.
    \item \textbf{Camera Support}: While professional testing uses heavy-duty support systems, the methodology accounts for typical tripod use and even carefully executed handheld shots.
\end{itemize}

\section{Statistical Reliability in DIY Context}

\subsection{Test-Retest Consistency}
The methodology emphasizes taking multiple test shots under slightly varying conditions to establish reliability ranges for measurements. This approach acknowledges the inherent variability of DIY testing while providing statistically meaningful results through aggregation.

This project aims to develop an integrated software solution that addresses these challenges through automated lens testing, direct camera control, and structured data management. The solution must be:

\begin{itemize}
    \item Accessible to photographers of varying technical expertise
    \item Capable of producing reliable, repeatable results
    \item Efficient in terms of time and resource requirements
    \item Comprehensive in its analysis of key lens characteristics
    \item Adaptable to different testing environments and equipment setups
\end{itemize}


\subsection{Comparison with Existing Solutions}

\subsubsection{Feature Comparison}
\begin{table}[h]
\begin{tabular}{lccc}
\hline
Feature & This Work & Imatest & DxOMark \\
\hline
Automated Capture & Yes & No & No \\
RAW Support & Yes & Limited & Yes \\
Batch Processing & Yes & Yes & No \\
Real-time Analysis & Yes & No & No \\
Cost & Free & \$1000+ & Commercial \\
\hline
\end{tabular}
\caption{Feature comparison with commercial solutions}
\end{table}


\subsection{Real-World Applications}

\subsubsection{Photography Studios}
Application benefits for professional use:
\begin{itemize}
    \item Automated lens testing workflow reduces time by 75%
    \item Consistent quality assessment across lens inventory
    \item Trackable lens performance over time
    \item Cost-effective alternative to commercial solutions
\end{itemize}

\subsubsection{Educational Settings}
Value in educational contexts:
\begin{itemize}
    \item Interactive learning tool for optical concepts
    \item Practical demonstration of lens characteristics
    \item Quantifiable results for student experiments
    \item Accessible platform for research projects
\end{itemize}

\subsection{Testing Tools and Standards}
\subsubsection{Test Charts and Patterns}
Modern lens testing relies on standardized test patterns:
\begin{itemize}
    \item ISO 12233 resolution charts
    \item Siemens star patterns for MTF measurement
    \item Grid patterns for distortion analysis
    \item Dot patterns for chromatic aberration testing
    \item Greyscale gradients for vignetting analysis
\end{itemize}

These tools together provide a comprehensive foundation for building sophisticated lens testing applications.


The selection and implementation of appropriate data formats significantly impacts system usability and interoperability.


\section{System Architecture}

The application follows a layered architecture pattern, with distinct components handling different aspects of the system. At its core, the application consists of three main layers: the camera interface layer, the analysis engine, and the user interface layer.

\subsection{Camera Interface Layer}

The camera interface layer provides abstraction over the physical camera hardware, utilizing the gphoto2 library for USB communication. This layer encapsulates all camera-specific operations, including:

\begin{itemize}
    \item Establishing and maintaining camera connections
    \item Capturing images in RAW format
    \item Retrieving camera settings and metadata
    \item Managing camera parameters such as aperture and shutter speed
\end{itemize}


%\begin{thebibliography}{9}
%\bibitem{gphoto2_compatible}
%GPhoto2. (n.d.). \textit{Supported Cameras}. Retrieved from \url{http://www.gphoto.org/proj/libgphoto2/support.php}
% [Add more references as needed]
%\end{thebibliography}
