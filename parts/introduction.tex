\section*{Background and motivation}
In today's rapidly advancing digital age of photography, the quality of a camera lens can significantly influence the outcome of photographic or videographic work. High-quality lenses can enhance sharpness, color accuracy, and overall image fidelity, while low-quality lenses can introduce undesirable artifacts such as distortion, chromatic aberration, and vignetting. Evaluating a lens's quality, therefore, becomes crucial for both amateur and professional photographers.

However, evaluating lens quality involves intricate techniques and specialized equipment, often making it inaccessible to many photographers. Current tools and methods, such as those provided by Imatest and Dxomark, offer comprehensive image quality assessments, but can be prohibitively expensive and complex for everyday users. These tools also often require controlled environments and advanced technical knowledge to operate effectively.

This thesis aims to bridge this gap by developing a user-friendly Python application that allows photographers to assess the quality of their lenses using at-home printable test charts and commonly available equipment. The application captures images of calibration elements, processes and analyzes these images to evaluate various lens properties, and provides a score (or a suite of scores) to describe the performance. This evaluation includes analyzing sharpness, vignetting, point spread function (PSF), and bokeh characteristics.


Evaluating lens quality accurately and conveniently is important for several reasons. For professionals, it can mean the difference between a good and a great shot, which can impact their reputation and business. For amateurs, understanding lens quality can enhance their learning curve and improve their photography skills. This application democratizes lens evaluation, making it accessible to a wider audience and helping photographers make informed decisions about their equipment.



The source code for this project is available on GitHub: \url{https://github.com/N4M3L355/clona}. A demonstration instance of the application is accessible at \url{https://clona.nameless.sk}. This instance runs in analysis-only mode through Docker containerization, allowing users to explore the interface and analysis capabilities using pre-existing datasets. While camera control features are disabled in this deployment, visitors can import their own datasets, load their own photos or examine sample lens tests to understand the application's functionality.

\section*{Problem statement}
In the field of photography and optical equipment evaluation, there is a significant gap between professional lens testing tools and accessible solutions for photographers and technicians. This gap presents several challenges:

%TODO: are all needed?
\begin{enumerate}
    \item \textbf{Integration Complexity}: Many existing solutions treat image capture and analysis as separate processes, requiring users to manually transfer files and manage different software tools. A unified solution that handles both capture and analysis would streamline the workflow.
    
    \item \textbf{Environmental Variability}: Lens testing results can vary significantly based on lighting conditions, camera settings, and test chart positioning. Methods are needed to ensure consistent results across different testing environments.
    
    \item \textbf{Data Management}: Photographers need structured ways to organize and compare lens test results over time, track performance across different focal lengths and apertures, and maintain historical data for their equipment.

    \item \textbf{Accessibility of Lens Testing}: Professional lens testing tools are costly, require specialized environments, and demand technical expertise, limiting their accessibility for nonexperts.
        
    \item \textbf{Technical Limitations}: DIY methods lack quantitative accuracy and reproducibility. Most open-source tools focus on isolated metrics like MTF or distortion, rather than comprehensive lens evaluation.
        
    \item \textbf{Data Management Challenges}: Existing tools lack systematic methods for organizing and comparing test results, maintaining consistent conditions, and tracking performance trends over time.

    \item \textbf{Automation and Workflow}: Manual testing is time-consuming, error-prone, and requires repeated adjustments. The absence of automated capture workflows limits efficiency.

    \item \textbf{Analysis and Interpretation}: Complex optical data requires expert interpretation. Tools for translating measurements into actionable insights and visual comparisons are insufficient.
\end{enumerate}

These challenges create a need for a solution that bridges the gap between professional testing equipment and accessible tools for photographers. This thesis proposes to address these challenges through the development of a comprehensive, user-friendly lens evaluation application that combines automated camera control, sophisticated image analysis, and intuitive result presentation.

% The key problems this thesis aims to solve include:
% \begin{itemize}
%     \item Designing and implementing an accessible yet accurate lens testing system
%     \item Automating the capture and analysis process while maintaining reliability
%     \item Presenting complex optical measurements in an understandable format
%     \item Ensuring consistent and reproducible results across different testing conditions
%     \item Managing and organizing lens testing data effectively
% \end{itemize}

By addressing these problems, this thesis contributes to making professional-grade lens testing more accessible to photographers, technicians, and educational institutions while maintaining scientific rigor and practical utility.

\section*{Objectives}
This thesis aims to develop an application that interfaces with cameras to capture RAW images and implements image processing algorithms for segmenting and preprocessing calibration elements within these images. The application then evaluates lens properties such as sharpness and vignetting with comprehensive scoring. A user-friendly web interface ensures accessibility for both experts and beginners. Features like session saving and consistent performance across various cameras and lighting conditions enhance usability and reliability.

\section*{Scope and limitations}

The scope of this thesis focuses on addressing key challenges in lens evaluation by developing a cost-effective and user-friendly tool that makes this process accessible to non-professionals. Robust image processing algorithms are employed to ensure accurate assessments of lens properties. The application is designed to deliver consistent results across various camera models and lighting conditions, showcasing its versatility. An intuitive web interface further enhances usability, catering to both advanced users and beginners. Additionally, the research explores designing a reliable and user-friendly application, determining effective image processing techniques for evaluating properties such as sharpness and bokeh, ensuring consistency across diverse setups, and assessing the practicality of using at-home printable test charts.

\section*{Thesis Structure}
This thesis is organized into five chapters. First chapter reviews existing tools and techniques for lens evaluation and explains key concepts and technologies. Second chapter describes the methodology, including the system architecture, data flow, and analysis scenarios. Third chapter describes the implementation process and core functionalities of the application. Fourth chapter presents the testing results, validation and comparisons with existing tools. The last chapter concludes with a summary of findings, limitations, and directions for future research.
