
\chapter{Technical Appendices}
\section{Source code}
The complete source code for this project is available on GitHub: \url{https://github.com/N4M3L355/clona} and in the attachment of this thesis in IS.

\section{Code Snippets}
\begin{spverbatim}
# Function to capture a photo and save it as a RAW file
def capture_photo_and_save(camera):
    file_path = camera.capture(gp.GP_CAPTURE_IMAGE)
    target = os.path.join(tempfile.gettempdir(), 
                          file_path.name)
    camera_file = camera.file_get(file_path.folder, 
                                  file_path.name, 
                                  gp.GP_FILE_TYPE_NORMAL)
    camera_file.save(target)
    return target

# Function to process a RAW image using rawpy
def process_raw_image(file_path):
    with rawpy.imread(file_path) as raw:
        rgb = raw.postprocess()
    return cv2.cvtColor(rgb, cv2.COLOR_RGB2BGR)

# Function to analyze sharpness using a Siemens star
def analyze_sharpness(image):
    gray = cv2.cvtColor(image, cv2.COLOR_BGR2GRAY)
    edges = cv2.Canny(gray, 50, 150)
    return edges

# Function to analyze vignetting using a white wall
def analyze_vignetting(image):
    gray = cv2.cvtColor(image, cv2.COLOR_BGR2GRAY)
    mean_intensity = np.mean(gray)
    return mean_intensity





# Function to capture a series of photos with different 
# aperture settings
def capture_series(camera, settings):
    file_paths = []
    config = camera.get_config(camera.context)
    aperture_cfg = config.get_child_by_name('aperture')

    for setting in settings:
        aperture_cfg.set_value(setting)
        camera.set_config(config, camera.context)
        file_path = camera.capture(gp.GP_CAPTURE_IMAGE)
        target = os.path.join(tempfile.gettempdir(), 
                              file_path.name)
        camera_file = camera.file_get(file_path.folder,
                                      file_path.name, 
                                      gp.GP_FILE_TYPE_NORMAL)
        camera_file.save(target)
        file_paths.append(target)
    return file_paths
\end{spverbatim}

\section{Test code}
Example of camera initialization test:
\begin{lstlisting}[language=Python]
def test_initialize_camera(mock_camera):
    """Test camera initialization"""
    with patch('gphoto2.Camera') as mock_camera_class:
        mock_camera_class.return_value = mock_camera
        mock_camera_class.autodetect = MagicMock(
            return_value=[(0, "Test Camera")])
        
        # Mock camera status check
        config = MagicMock()
        widget = MagicMock()
        widget.get_value.return_value = 0
        config.get_child_by_name.return_value = widget
        mock_camera.get_config.return_value = config
        
        manager = CameraManager()
        success = manager.initialize_camera()
        
        assert success
        assert manager.connected
        assert manager.camera is not None
\end{lstlisting}
\\
Example of photo capture test:
\begin{lstlisting}[language=Python]
def test_handle_capture_photo(ui_instance, temp_dir):
    """Test photo capture handling"""
    capture_result = {
        'path': os.path.join(temp_dir, "test.jpg"),
        'metadata': {
            'camera_settings': {
                'aperture': '1.8',
                'shutterspeed': '1/100',
                'iso': '400'
            }
        }
    }
    
    ui_instance.camera_manager.capture_image = MagicMock(
        return_value=capture_result)
    ui_instance.camera_manager.connected = True
    
    with patch('nicegui.ui.notify') as mock_notify:
        ui_instance.handle_capture_photo()
        ui_instance.camera_manager.capture_image\
            .assert_called_once()
        mock_notify.assert_called_once()
\end{lstlisting}

%\section{Test Charts and Patterns}

% todo